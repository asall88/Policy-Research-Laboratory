\documentclass[11pt]{article}

% Preamble!!
\usepackage{titling}
\setlength{\voffset}{0.7in}
\setlength{\droptitle}{-10em}
\usepackage{titlesec}
\titlelabel{\thetitle.\quad}
\usepackage{xcolor}
\usepackage{adjustbox}
\usepackage{amsmath, amsthm, amsfonts, amssymb}
\usepackage{graphicx}
\usepackage{setspace}
\usepackage{longtable}
\usepackage{breqn}
\usepackage{lscape}
\usepackage{indentfirst}
\usepackage[labelsep=period,justification=justified,singlelinecheck=false,font = footnotesize, labelfont=bf]{caption}
\usepackage{booktabs}
\usepackage{tabularx,ragged2e}
\usepackage{natbib}
\usepackage{rotating}
\usepackage{placeins}
\usepackage{subcaption}
\usepackage{hyperref}
\definecolor{burntorange}{rgb}{0.8, 0.33, 0.0}
\hypersetup{
    colorlinks=true,
    linkcolor=orange,
    filecolor=magenta,      
    urlcolor=burntorange,
}
\pagenumbering{arabic}

\usepackage{bbm}
\usepackage[margin=1in]{geometry}

\usepackage{enumerate}
\usepackage{array}
\usepackage[T1]{fontenc}
\usepackage[font=small,labelfont=bf,tableposition=top]{caption}
\usepackage{mathtools}
\newcommand\eho{\stackrel{\mathclap{\small\mbox{$H_0$}}}{=}}
\newcommand\sho{\stackrel{\mathclap{\small\mbox{$*$}}}{=}}
\newcommand\dho{\stackrel{\mathclap{\small\mbox{$d$}}}{=}}
\newcommand\qho{\stackrel{\mathclap{\small\mbox{$?$}}}{=}}

% \usepackage{authblk}


\DeclareCaptionLabelFormat{andtable}{#1~#2  \&  \tablename~\thetable}

\usepackage{fullpage, amsmath, amssymb, amsthm, bbm, color}
\usepackage{graphicx,caption,subcaption,placeins}
\usepackage{dsfont}

\usepackage{natbib}
\bibpunct{(}{)}{;}{a}{,}{,}

\newtheorem{assumption}{Assumption}
\newtheorem{definition}{Definition}
\newtheorem{theorem}{Theorem}
\newtheorem{example}{Example}
\newtheorem{procedure}{Procedure}


\usepackage{tikz}
\usetikzlibrary{positioning,chains,fit,shapes,calc}

\definecolor{myblue}{RGB}{80,80,160}
\definecolor{mygreen}{RGB}{80,160,80}


\begin{document}

\title{FIN 373 Homework 2 \\ {\large due: \textbf{9/7/21}}}
\date{}
\maketitle

\vspace{-20mm}

\noindent Instructions: Please submit solutions on canvas.  Only a knitted pdf of an {\tt Rmarkdown} file will be accepted.
\\

\noindent \textbf{Problem 1:} QSS exercise 1.5.1.  Surveys are frequently used to measure political behavior such as
voter turnout, but some researchers are concerned about the accuracy
of self-reports.  In particular, they worry about possible \textit{social
desirability bias} where in post-election surveys, respondents who did
not vote in an election lie about not having voted because they may
feel that they should have voted.  Is such a bias present in the
American National Election Studies (ANES)?  The ANES is a nation-wide
survey that has been conducted for every election since 1948.  The
ANES conducts face-to-face interviews with a nationally representative
sample of adults.  The table below displays the names and descriptions
of variables in the {\tt turnout.csv} data file.
\vspace{3mm}
\begin{center}
\begin{tabular}{l p{9cm}}
 \hline
\textit{Variable} & \textit{Description} \\
\hline
 \verb year &               Election year \\
\verb ANES &               ANES estimated turnout (percentage) \\
\verb VEP &                Voting Eligible Population (in thousands) \\
\verb VAP &                Voting Age Population (in thousands) \\
\verb total &              Total ballots cast for highest office (in thousands) \\
\verb felons & Total ineligible felons (in thousands) \\
\verb noncitizens &  Total non-citizens (in thousands) \\
\verb overseas &           Total eligible overseas voters (in thousands) \\
\verb osvoters &           Total ballots counted by overseas voters (in thousands)\\
\hline
\end{tabular}
\end{center}
\vspace{2mm}
\begin{enumerate}[a.]
	\item Load the data into {\tt R} and check the dimensions of the data.
  Also, obtain a summary of the data.  How many observations are
  there?  What is the range of years covered in this data set?
  \item Calculate the turnout rate based on the voting age population or
  VAP. Note that for this data set, we must add the total number of
  eligible overseas voters since the VAP variable does not
  include these individuals in the count. Next, calculate the turnout
  rate using the voting eligible population or VEP.  What difference
  do you observe?
  \item Compute the difference between VAP and ANES estimates of turnout
  rate.  How big is the difference on average?  What is the range of
  the difference?  Conduct the same comparison for the VEP and ANES
  estimates of voter turnout.  Briefly comment on the results.
  \item Compare the VEP turnout rate with the ANES turnout rate
  separately for presidential elections and midterm elections.  Note
  that the data set excludes the year 2006. Does the bias of the ANES vary across election types?
  \item Divide the data into half by election years such that you subset
  the data into two periods.  Calculate the difference between the VEP
  turnout rate and the ANES turnout rate separately for each period.  Has the bias of the ANES increased over time?
  \item The ANES does not interview overseas voters and
  prisoners. Calculate an adjustment to the 2008 VAP turnout
  rate. Begin by subtracting the total number of ineligible felons and
  non-citizens from the VAP to calculate an adjusted VAP. Next,
  calculate an adjusted VAP turnout rate, taking care to subtract the
  number of overseas ballots counted from the total ballots in 2008.
  Compare the adjusted VAP turnout with the unadjusted VAP, VEP, and
  the ANES turnout rate. Briefly discuss the results.
\end{enumerate}

\vspace{7mm}
\noindent \textbf{Problem 2:} QSS exercise 2.8.1.  The STAR (Student-Teacher Achievement Ratio) Project is a four year
longitudinal study examining the effect of class size in early
grade levels on educational performance and personal
development.\footnote{This exercise is in part based on:
 Mosteller, Frederick. 1997. \href{http://dx.doi.org/10.2307/3824562}{The Tennessee Study of Class Size in the 
 Early School Grades.} \textit{Bulletin of 
 the American Academy of Arts and Sciences} 50(7): 14-25.}
  
A longitudinal study is one in which the same
participants are followed over time.  This particular study lasted
from 1985 to 1989 involved 11,601 students. During the four years of
the study, students were randomly assigned to small classes,
regular-sized classes, or regular-sized classes with an aide.  In all,
the experiment cost around \$12 million. Even though the program
stopped in 1989 after the first kindergarten class in the program
finished third grade, collection of various measurements (e.g.,
performance on tests in eighth grade, overall high school GPA)
continued through the end of participants' high school attendance.

We will analyze just a portion of this data to investigate whether the
small class sizes improved performance or not. The data file name is
{\tt STAR.csv}.  The names and
descriptions of variables in this data set are:
\vspace{3mm}
\begin{center}
\begin{tabular}{l p{9cm}}
 \hline
\textit{Variable} & \textit{Description} \\
\hline
 \verb race &               Student's race (White = 1, Black = 2, Asian = 3,
                      Hispanic = 4,  Native American = 5, Others = 6)\\
 \verb classtype &          Type of kindergarten class (small = 1, regular = 2, regular with aide = 3)\\
 \verb g4math &             Total scaled score for math portion of fourth grade standardized test \\
 \verb g4reading &          Total scaled score for reading portion of fourth grade standardized test \\
 \verb yearssmall &         Number of years in small classes \\
 \verb hsgrad &             High school graduation (did graduate = 1, did not graduate = 0)\\
\hline
\end{tabular}
\end{center}
\vspace{2mm}
\begin{enumerate}[a.]
\item Create a new factor variable called {\tt kinder} in the data
  frame.  This variable should recode {\tt classtype} by changing
  integer values to their corresponding informative labels (e.g.,
  change 1 to small etc.).  Similarly, recode the
  {\tt race} variable into a factor variable with four levels
  (White, Black, Hispanic, Others) by
  combining Asians and Native Americans as the Others
  category.  For the {\tt race} variable, overwrite the original
  variable in the data frame rather than creating a new one.  
  \item How does performance on fourth grade reading and math tests for
  those students assigned to a small class in kindergarten compare
  with those assigned to a regular-sized class?  Do students in the
  smaller classes perform better?  Use means to make this comparison
  while removing missing values.  Give a brief substantive
  interpretation of the results.  To understand the size of the
  estimated effects, compare them with the standard deviation of the
  test scores.  Recall
  that {\tt na.rm = TRUE} can be added to functions in order to
  remove missing data (see section 1.3.5).
\item Instead of comparing just average scores of reading and math
  tests between those students assigned to small classes and those
  assigned to regular-sized classes, look at the entire range of
  possible scores.  To do so, compare a high score, defined as the
  66th percentile, and a low score (the 33rd percentile) for small
  classes with the corresponding score for regular classes.  These are
  examples of \textit{quantile treatment effects}.  Does this analysis
  add anything to the analysis based on mean in the previous question?
  \item Some students were in small classes for all four years that the
  STAR program ran. Others were assigned to small classes for only one
  year and had either regular classes or regular classes with an aide
  for the rest. How many such students of each type are in the data
  set?  Create a contingency table of proportions using the
  {\tt kinder} and {\tt yearsmall} variables.  Does participation
  in more years of small classes make a greater difference in test
  scores?  Compare the average and median reading and math test scores
  across students who spent different numbers of years in small
  classes.
  \item Examine whether the STAR program reduced the achievement gaps
  across different racial groups.  Begin by comparing the average
  reading and math test scores between white and minority students
  (i.e., Blacks and Hispanics) among those students who were assigned
  to regular classes with no aide.  Conduct the same comparison among
  those students who were assigned to small classes.  Give a brief
  substantive interpretation of the results of your analysis.
  \item Consider the long term effects of kindergarten class size.
  Compare high school graduation rates across students assigned to
  different class types.  Also, examine whether graduation rates
  differ by the number of years spent in small classes.  Finally, as
  done in the previous question, investigate whether the STAR program
  has reduced the racial gap between white and minority students'
  graduation rates.  Briefly discuss the results.
\end{enumerate}	

\vspace{7mm}
\noindent \textbf{Problem 3:} First, describe in your own words the importance of randomization, or random assignment of a treatment, for accurately estimating a causal effect. Second, why is it wrong to estimate the causal effect of COVID lockdowns as the difference in average outcomes for ``lockdown states'' (states that implemented a lockdown) and ``non-lockdown states'' (states that did not implement a lockdown?



\end{document}

